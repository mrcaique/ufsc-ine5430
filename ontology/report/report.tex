\documentclass{article}
\usepackage[utf8]{inputenc}
\usepackage[brazil]{babel} % Texto em português do Brasil
\usepackage[a4paper, left=20mm, right=20mm, top=20mm, bottom=20mm]{geometry}
\usepackage[num]{abntex2cite}
\usepackage{csquotes}

\title{\textbf{INE5430 - Inteligência Artificial \\
        \large Trabalho T2 - Ontologias}}
\author{
    Caique Rodrigues Marques \\
    {\texttt{c.r.marques@grad.ufsc.br}}
    \and
    Fernando Jorge Mota \\
    {\texttt{contato@fjorgemota.com}}
    \vspace{-5mm}
}
\date{}
\begin{document}
    \maketitle
    \section*{Domínio}
        O tema para a ontologia deste trabalho é baseado na série \textit{O
        Guia do Mochileiro das Galáxias} (1979), por Douglas Adams, dividido-se
        numa trilogia de seis livros - descrição dada pelo autor. A obra
        apresenta as aventuras de Arthur Dent, um pacato inglês, e de seu amigo
        Ford Prefect, um nativo de Beetelgeuse que vive disfarçado na Terra
        enquanto faz a pesquisa para escrever a nova edição do Guia do
        Mochileiro das Galáxias. A Terra é demolida por vogons para a
        construção de um desvio intergaláctico, porém, a dupla escapa pegando
        carona numa nave alienígena graças a Ford, desenrolando uma série de
        eventos inesperados. O grande destaque do livro é o humor inteligente,
        Adams fez questão de apresentar seus personagens em situações mais
        esquisitas possíveis, satirizando e tirando deboche de diversos temas
        como política e religião.
        
        \subsection*{A ontologia}
            O universo de \textit{O Guia do Mochileiro das Galáxias} é bem
            extenso e vasto, portanto, nos delimitamos a pegar apenas uma parte
            da série. A divisão principal está em três grandes classes: vida
            (\textit{life}), Universo (\textit{universe}) e tudo mais
            (\textit{owl:Thing}), onde o primeiro corresponde aos seres vivos,
            como humanos, vogons e golfinhos; o segundo corresponde a corpos
            celestes, que são planetas e sistemas solares; por fim, o terceiro
            corresponde ao resto, no caso, ferramentas e artes.
            
            As propriedades são mais variadas, correspondendo mais a ações
            efetuadas por seres vivos, como comer, produzir, viajar, pegar
            carona, gostar, odiar, etc., e o inverso de algumas também estão
            inclusas, como ser comido, ser produzido, ser destruído, etc..
            Outras propriedades também são inversíveis, como Arthur ser amigo
            de Ford, implica que Ford é amigo de Arthur.
            
        \subsection*{Testes}
            Cada classe tem suas propriedades, que acabam sendo aplicadas às
            suas subclasses (quando têm) e às suas instâncias.
            
            Em questão de cardinalidade exata, a máquina consegue achar
            facilmente uma inconsistência quando algo não é do valor exato
            definido na equivalência. Por exemplo, a classe \texttt{life} tem
            como equivalência que é provinda de um e apenas um planeta,
            portanto, um dado ser vivo é nativo de apenas um planeta, assim as
            subclasses de \texttt{life} também vão ter tal equivalência. Logo,
            ao especificar que humanos são nativos do planeta Terra, a máquina
            conseguirá inferir que qualquer instância de humano será do planeta
            Terra e de nenhum outro lugar.
            
            A cardinalidade mínima indica que tem de haver no mínimo uma
            condição para acontecer, portanto, a vida só pode existir se ela
            for provinda de algum universo.
            
            As relações inversas são definidas de princípio, como
            \texttt{has\_pet} é o inverso de \texttt{is\_pet\_of}, assim, se
            Benjy e Frankie são mascotes da Trillian, a máquina conseguirá
            inferir que Trillian tem Benjy e Frankie como mascotes.
            
            As relações simétricas definem que se \texttt{A} tem relação com
            \texttt{B}, então \texttt{B} também terá relação com \texttt{A}. Um
            exemplo de relação simétrica é parentesco, pois se Ford é parente
            de Zaphod, então a máquina consegue inferir que Zaphod é, também,
            parente de Ford.

    \section*{SubClassOf e EquivalentTo}
        \begin{itemize}
            \item Na linguagem OWL, quando a classe \texttt{A} é subclasse da
                propriedade \texttt{B}, isto significa que a classe \texttt{A}
                herda tal propriedade, porém, se uma instância tiver a
                propriedade \texttt{B}, não necessariamente será do tipo
                \texttt{A}. Por exemplo, as ferramentas (\textit{tools}) têm
                como subclasses robôs (\textit{robot}) e naves espaciais
                (\textit{spaceship}), portanto, qualquer robô ou nave sempre
                será produzido por algum vogon e/ou por algum humano. Se o Guia
                do Mochileiro das Galáxias é produzido por algum ser vivo, não
                necessariamente ele será uma ferramenta ou um robô, mas será um
                livro. 
            
            \item A equivalência é uma relação bidirecional, ou seja, se a
                classe \texttt{A} é equivalente à propriedade \texttt{B}, então
                significa que quaisquer que sejam as instâncias que tenham a
                propriedade \texttt{B}, ela será do tipo \texttt{A}. Por
                exemplo, a classe \texttt{human} tem como equivalência que
                seres deste tipo são nativos do planeta Terra, então, se Arthur
                Dent é humano, claramente ele é do planeta Terra.
        \end{itemize}
    
    \section*{Sobre o motor de inferência}
        O motor de inferência utilizado neste trabalho foi o \textbf{HermiT},
        conforme sugestão dada pelo professor durante a aula. O HermiT é um
        motor de inferência para ontologias usando a OWL, de código aberto e
        licenciado sob a GPL, que é capaz de inferir variados aspectos a
        respeito de uma ontologia, como se ela é consistente ou não, por
        exemplo.
        
        Entre os destaques a respeito deste motor de inferência, se encontra o
        fato de que o HermiT usa uma forma mais avançada de cálculo, denominada
        \textit{"hypertableau calculus"}, que permite que o motor faça em
        segundos, inferências que antes levavam minutos ou até horas, se
        tornando o primeiro motor de inferência capaz de classificar ontologias
        antes consideradas muito complexas para serem trabalhadas.
        
        Quanto a seu funcionamento, o HermiT funciona sob um princípio inovador
        quando comparado com outros motores de inferência. Primeiramente, para
        uma determinada ontologia, é criado um "\textit{description graph}"
        (grafo de descrição, em tradução livre), que permite a descrição dos
        objetos estruturados que compõem a OWL de uma forma bastante simples e
        precisa, o que resolve o problema do OWL em descrever estruturas
        arbitrariamente conectadas e assim evita problemas de performance
        durante a realização das inferências. 
        
        Depois disso, a inferência é realizada em duas etapas: uma de
        pré-processamento, no qual os axiomas e o grafo de descrição são
        traduzidos em um conjunto equisatisfazível de regras (ou seja, um
        conjunto no qual se uma regra é satisfazível, então a outra também o é
        e vice-versa), e outra denominada "\textit{hypertableau}", no qual se
        busca construir um modelo baseando-se em um conjunto de regras
        relacionada ao OWL e aos resultados do pré-processamento.

    \nocite{stackoverflow, hermit, mgs08structured-objects}

    \bibliography{references}
\end{document}
