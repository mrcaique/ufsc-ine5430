\documentclass{article}
\usepackage[utf8]{inputenc}
\usepackage[brazil]{babel} % Texto em português do Brasil
\usepackage[a4paper, left=20mm, right=20mm, top=20mm, bottom=20mm]{geometry}
\usepackage[colorlinks, urlcolor=blue, citecolor=red]{hyperref}
\usepackage{mathtools}

\title{\textbf{INE5430 - Inteligência Artificial \\
        \large Trabalho T1}}
\author{
    Caique Rodrigues Marques \\
    {\texttt{c.r.marques@grad.ufsc.br}}
    \and
    Fernando Jorge Mota \\
    {\texttt{contato@fjorgemota.com}}
    \vspace{-5mm}
}
\date{20 de setembro de 2016}

\begin{document}
    
    \maketitle
    
    \section*{Especificações Técnicas}
        \begin{itemize}
            \item Escrito na linguagem de programação Python
            \item Testado o funcionamento em Python 2.7 e em Python 3.5
            \item Para melhor execução, recomenda-se o uso do
                \href{http://pypy.org/}{PyPy}
            \item Para executar o programa, vá para a pasta \texttt{gomoku/} e
                rode o comando: \texttt{python program.py}
        \end{itemize}
    
    \section*{Estrutura de Dados}
        A estrutura de dados usada no programa, definida na classe
        \texttt{state.py}, é imutável. Bastante similar com um grafo - mais
        especificadamente, uma árvore - cada objeto dessa estrutura possui uma
        série de atributos que são usados durante a execução do programa:
        
        \begin{itemize}
            \item Uma matriz representando o tabuleiro vazio - preenchido com
                "+" internamente, para fins de representação;
            \item Uma string representando o jogador que está jogando naquele
                estado - inicialmente, pelas regras do jogo, será o jogador
                "O", que representa a peça vermelha;
            \item Uma string preparada para armazenar uma mensagem a ser
                exibida para o usuário. Usada principalmente em casos de erro
                ou avisos;
            \item Um objeto da classe \texttt{Sequences} representando as
                sequências formadas; 
            \item Cada movimento, que é um objeto da classe \texttt{Move}, é
                armazenado em \texttt{Sequences}; 
            \item Um ponteiro para o estado que gerou originalmente o estado
                que se está usando.
        \end{itemize}
        
        A partir de um determinado estado, então, essa estrutura cria cópias de
        si mesma com apenas alguns atributos modificados e um ponteiro para a
        versão que a criou. Quando é feita uma jogada, por exemplo, um novo
        estado é criado modificando apenas a matriz e o ponteiro "pai"
        apontando para o estado no qual foi feito aquela jogada. Da mesma
        forma, quando é imprimida uma mensagem, é criado um novo estado apenas
        configurando a nova mensagem e o ponteiro "pai" correspondente.
        
        No final das contas, temos, nessa estrutura, algo bem parecido com uma
        árvore/grafo: o estado inicial seria a raiz e nenhum nodo filho seria,
        a principio, gerado sem que houvesse extrema necessidade. Dessa forma,
        é possível evitar uso desnecessário de memória e, pelo fato de cada
        objeto ser imutável, temos aqui uma vantagem: cada novo objeto só
        modifica os atributos modificados naquela operação. Todos os atributos
        que não são efetivamente modificados são referenciados por ponteiros no
        novo objeto criado, ajudando a economizar memória visto que a estrutura
        realiza cópias de todo atributo a cada operação feita.
        
        Para fins de economia de processamento com o algoritmo de busca, o
        minimax, o estado inicial gera uma jogada mais ao centro e em seus
        filhos as jogadas continuam mais concentradas no centro do tabuleiro,
        que é onde acontece grande parte do jogo. Isto serve como uma
        otimização da busca evitando que os estados desnecessários (jogadas
        iniciais nas bordas, por exemplo) não sejam geradas de princípio.
    
    \section*{Função Heurística}
        \subsection*{Considerações}
        Para uma máquina que queira jogar o jogo Gomoku é necessário que ela
        tenha uma boa estratégia a fim de conseguir garantir a vitória. A
        principal estratégia a se usar é considerar as inúmeras possibilidades
        de jogadas em uma determinada partida, no entanto, é inviável para uma
        máquina conseguir computar as várias jogadas e suas consequências
        possíveis, considerando o tamanho do tabuleiro e a quantidade de
        jogadas possíveis.
        
        Uma heurística é determinada para que a máquina consiga melhores
        estratégias a partir de uma estimativa, que diz ao computador, a partir
        de um determinado estado, se aquele estado está mais próximo de uma
        vitória ou de uma derrota para o sistema.
        
        A máquina pode, portanto, trabalhar mais na ofensiva ou mais na
        defensiva, dependendo de como estiver a sua situação no jogo. Os
        seguintes aspectos do jogo serão considerados pelo nosso algoritmo para
        uma boa estratégia de jogada ofensiva:
        
        \begin{itemize}
            \item O número de peças usadas e, portanto, o quanto o jogo está
                próximo do fim;
            \item A quantidade de duplas, triplas, quádruplas ou quíntuplas
                formadas pelo computador em uma jogada;
            \item Uma jogada que forma mais sequências contíguas também
                favorecem a vitória. Por exemplo, quando as peças estão
                dispostas onde forma um "L", tem duas sequências que é uma na
                horizontal e uma na vertical, aumentando portanto as chances de
                vitória.
        \end{itemize}
        
        As seguintes situações são consideradas para uma jogada mais defensiva:
    
        \begin{itemize}
            \item Impedir o adversário de formar quíntuplas, quádruplas ou
                triplas;
            \item Considerar uma jogada em que evite que o adversário consiga
                formar sequências contíguas de peças, diminuindo as chances
                dele de conseguir, por exemplo, de formar duas triplas em forma
                de "L".
        \end{itemize}
        
        Ainda tem as ações que dê vantagens à máquina:
        \begin{itemize}
            \item Colocar as peças onde há mais possibilidades de formar
                sequências, por exemplo, colocar um peça mais próxima ao centro
                dá mais chances de sequência do que colocar nas bordas;
            \item Verificar quantas sequências já foram montadas (duplas,
                triplas e quádruplas) para uma possível quíntupla;
            \item Avaliar as sequências já formadas e ver se é possível na
                jogada em questão fechar uma sequência maior a partir de duas
                menores, por exemplo, duas duplas separadas apenas por uma
                casa, ao colocar uma peça no meio da dupla, forma uma
                quíntupla;
            \item A máquina também deve avaliar em quantas casas estão
                separadas duas fileiras de peças de mesma cor;
            \item O número de jogadas realizadas até o término no jogo.
        \end{itemize}
        
        Dadas todas as situações listadas, é possível atribuir uma determinada
        prioridade a cada uma, de forma que a máquina saiba que o adversário
        formando uma tripla é razoavelmente mais prejudicial à sua jogada do
        que a máquina formar uma dupla. A saída de uma função heurística
        considerando os aspectos acima com um peso pré-determinado para cada
        uma define o quanto a máquina está ganhando ou perdendo no jogo.
        
        \subsection*{Implementação}
        A análise da heurística é feita considerando diversos aspectos para uma
        boa jogada. Em \texttt{heuristic.py} é definido uma série de funções
        que avaliam possíveis formações de sequências e atribui pesos a cada
        uma e a função geral \texttt{evaluate()}, que recebe um estado e o
        jogador atual, retorna as conclusões tiradas das jogadas realizadas.
        
        A seguinte função $t(s)$, onde $s$ corresponde a uma sequência, define
        como é a análise do tabuleiro pela máquina para tirar uma conclusão:
        \begin{gather*}
            t(s) = length(s) \times block(s) \times merge(s)
        \end{gather*}
        Onde há três funções: \texttt{length()} avalia o tamanho da sequência
        (quanto maior, melhor a pontuação final), \texttt{block()} avalia a
        abertura de uma sequência (mais pontos se estiver aberta em ambas as
        direções) e \texttt{merge()} avalia a junção de sequências (aumenta a
        pontuação se é possível juntar sequências em ambos os lados de uma
        sequência).
        
        Depois de conseguir a saída da função $t(s)$ referente a todas as
        sequências do tabuleiro, tanto para o jogador ($sp$), quanto para o
        adversário ($sa$), a seguinte operação é feita:
        \begin{gather*}
            result = \frac{sp}{nsp} - \frac{sa}{nsa}
        \end{gather*}
        Onde $nsp$ e $nsa$ correspondem ao número de sequências do jogador e do
        adversário, respectivamente. A conclusão é definida em $result$.
        
    \section*{Função Utilidade}
        A função de utilidade define uma certeza, que está próximo do final do
        jogo e ela define qual a melhor estratégia tomar para terminar o jogo
        da melhor maneira possível. É necessário que uma função de utilidade
        faça um balanço de tudo o que foi feito para, então, atribuir um peso
        final para a melhor decisão.
        
        Portanto, a seguinte função descreve a utilidade:
        \begin{gather*}
            utlty = \frac{score}{jgds} \times x 
        \end{gather*}
        Onde $score$ corresponde a uma pontuação dada ao jogador que formou
        mais sequências que o adversário (maior se fez mais que o adversário) e
        $jgds$ corresponde ao número de peças que o jogador usou. Tudo isso é
        multiplicado pelo valor de $x$ que define se aconteceu uma vitória ou
        uma derrota ou um empate. No caso de vitórias, o valor é positivo ($x =
        +1$), derrota é negativo ($x = -1$) e em empate é neutro ($x = 0$).
        
    \section*{Detecção de sequências}
        Em cada jogada no Gomoku, representando um estado, uma jogada é um
        objeto da classe \texttt{Move}, nela está a posição $x$ e $y$ da peça
        no tabuleiro e se foi o jogador \texttt{X} ou \texttt{O} que a jogou. A
        cada nova jogada, o objeto \texttt{Move} é armazenado no objeto da
        classe \texttt{Sequences}, portanto, ela conterá todas as jogadas
        realizadas e dela é possível achar as sequências sem a necessidade de
        percorrer o tabuleiro todo. Ambas as estruturas, \texttt{Sequences} e
        \texttt{Move} são imutáveis.
        
        Contendo um objeto imutável com todas as jogadas realizadas, a classe
        \texttt{Sequences.py} apresenta que é possível coletar as sequências de
        acordo com uma especificação: coletar sequências de um jogador, coletar
        sequências pelo tamanho, coletar sequências que estão fechadas, etc..
        
        Referente a uma sequência em específico, a classe \texttt{Sequence}
        torna a análise possível. Assim, é viável avaliar as extremidades de
        uma sequência (se ela possui as extremidades inferior e superior
        abertas ou fechadas) e a menor distância de uma extremidade da
        sequência a uma peça adversária. 
        
        Baseando-se na distância de uma peça a outra, é possível realizar
        junções (\texttt{merges}) de sequências de peças de mesma cor, por
        exemplo: dada uma disposição (horizontal, vertical ou diagonal), as
        duas extremidades da sequência são marcados, ao encontrar uma outra
        sequência na mesma disposição, suas extremidades também são marcadas.
        Se a posição de uma extremidade de uma sequência for igual à posição de
        uma extremidade oposta de uma outra sequência, então existe a
        possibilidade de junção de sequências; considerando que não haja peças
        adversárias no meio do caminho entre as duas sequências.
        
    \section*{Detecção de vitória}
        Conforme especificado na seção anterior, é realizado uma busca no
        objeto \texttt{Sequences} para avaliar as sequências de peças e quem as
        jogou. Assim, bastando procurar por sequências onde o tamanho é maior
        que cinco, já se detecta que o jogo terminou e um vencedor foi
        declarado.
        
    \section*{Problemas encontrados}
        Apesar das modificações realizadas, o algoritmo não ficou livre de
        problemas. O mais notável é a questão da performance, mesmo usando o
        PyPy, não chega a ter o tempo de resposta desejável para uma
        inteligência artificial. Relacionado a isso, também se notou ser
        difícil fazer com que a máquina avance mais níveis da árvore de
        jogadas, sendo mais viável ficar com uma profundidade relativamente
        baixa de busca, de apenas 2 níveis.

\end{document}
