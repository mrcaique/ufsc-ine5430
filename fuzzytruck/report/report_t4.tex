\documentclass{article}
\usepackage[utf8]{inputenc}
\usepackage[brazil]{babel}
\usepackage[a4paper, left=20mm, right=20mm, top=20mm, bottom=20mm]{geometry}
\usepackage[fleqn]{amsmath}

\title{\textbf{INE5430 - Inteligência Artificial \\
        \large Trabalho T4 - Sistemas Fuzzy}}
\author{
    Caique Rodrigues Marques \\
    {\texttt{c.r.marques@grad.ufsc.br}}
    \and
    Fernando Jorge Mota \\
    {\texttt{contato@fjorgemota.com}}
    \vspace{-5mm}
}
\date{}
\begin{document}
    \maketitle
    
    \section*{Introdução}
        A base para a execução deste trabalho está no uso de conceitos do
        sistema \textit{fuzzy} ou sistema nebuloso, composto de lógica
        \textit{fuzzy} e de conjunto \textit{fuzzy}. O estudo de conjuntos
        \textit{fuzzy} foi introduzido por Lofti Zadeh, em 1965, e é uma
        extensão da teoria de conjuntos clássica. A lógica \textit{fuzzy} é uma
        lógica multivalorada onde a informação, ao contrário da lógica
        booleana, não é atribuída a um ou zero, mas podendo estar entre esses
        dois valores.
        
        O cerne do raciocínio \textit{fuzzy} é da transformação de informações
        linguísticas (podendo ser dúbias e vagas) em valores numéricos para que
        possam ser trabalhadas matematicamente e computacionalmente, depois, é
        realizado o inverso para que as soluções sejam avaliadas.
    
    \section*{O sistema}
        O sistema utilizado consiste em fazer um caminhão estacionar na vaga
        especificada, em marcha ré, usando uma série de regras \textit{fuzzy}.
        As variáveis que devem ser consideradas são: a posição $x$ e $y$ do
        caminhão; a direção no qual ele está e o ângulo em que o volante está
        virado.
        
    \subsection*{Variáveis do sistema}
        As variáveis entrada do sistema são valores $x$ e $y$, apontando a
        posição inicial do caminhão, e a direção, que consiste onde a frente do
        caminhão está apontado (este dado é obtido a partir da conversão do
        angulo atual no qual o caminhão se encontra). A partir desses dados, um
        ângulo é gerado como saída, mostrando qual o angulo do volante do
        caminhão deve ser modificado antes de executar o próximo movimento a
        ré, num intervalo que vai de -30$^\circ$ até 30$^\circ$. Outro ponto a
        notar é a delimitação das variáveis: os eixos $x$ e $y$ estão limitados
        ao intervalo $(0, 1)$; as direções envolvem cima, baixo, esquerda e
        direita e são definidas a partir de seus valores respectivos em uma
        escala de 0$^\circ$ a 360$^\circ$; Já o ângulo é delimitado de
        -30$^\circ$ até 30$^\circ$, embora seja enviado para o servidor como um
        intervalo que vai de -1 a 1. Como esses intervalos fazem parte do
        funcionamento do sistema, vamos especificar e listar as condições que
        classificam um determinado valor fornecido pelo servidor dado em um
        termo que pode ser usado numa regra de forma mais simples. Segue a
        lista:
        \begin{itemize}
            \item \textbf{X}
                \begin{itemize}
                    \item \texttt{\textbf{tooLeft}} - Intervalo entre 0 e 0.3;
                    \item \texttt{\textbf{left}} - Intervalo entre 0.1 e 0.5;
                    \item \texttt{\textbf{half}} - Intervalo entre 0.4 e 0.6;
                    \item \texttt{\textbf{right}} - Intervalo entre 0.5 e 0.9;
                    \item \texttt{\textbf{tooRight}} - Intervalo entre 0.7 e 1.
                \end{itemize}
            
            \item Y 
                \begin{itemize}
                    \item \texttt{\textbf{up}} - Intervalo entre 0 e 0.8;
                    \item \texttt{\textbf{bottom}} - Intervalo entre 0.5 e 1.
                \end{itemize}

            \item Direção (\texttt{direction}, no código)
                \begin{itemize}
                    \item \texttt{\textbf{left}} - Intervalo entre 90 e 270;
                    \item \texttt{\textbf{right}} - Intervalo entre 0 e 90 e
                        intervalo entre 270 e 360;
                    \item \texttt{\textbf{up}} - Intervalo entre 0 e 180;
                    \item \texttt{\textbf{down}} - Intervalo entre 180 e 360.
                \end{itemize}
        \end{itemize}
        
        Além dos termos acima, também definimos um termo que é usado como
        retorno para cada regra e que respeita intervalos similares ao do termo
        X:
        
        \begin{itemize}
            \item Ângulo (\texttt{angle}, no código)
            \begin{itemize}
                \item \texttt{\textbf{tooLeft}} - Intervalo entre -1 e -0.4;
                \item \texttt{\textbf{left}} - Intervalo entre -0.8 e -0.2;
                \item \texttt{\textbf{center}} - Intervalo entre -0.4 a 0.4;
                \item \texttt{\textbf{right}} - Intervalo entre 0.2 e 0.8;
                \item \texttt{\textbf{tooRight}} - Intervalo entre 0.4 a 1.
            \end{itemize}
        \end{itemize}
    
    \subsection*{Regras}
        A partir das definições especificadas na seção anterior, uma série de
        40 regras são definidas para determinar qual caminho o caminhão deve
        percorrer. Todas as regras são condicionais (\texttt{IF, THEN}), sendo
        que dados os $x$, $y$ e a direção, então o ângulo do volante deve ser
        uma das cinco posições especificadas na listagem anterior e são
        aplicados cálculos de lógica \textit{fuzzy} para encontrar o valor que
        será retornado. Alguns exemplos dessas regras são encontrados abaixo:
        
        \begin{center}
            \begin{tabular}{c c c c c}
                \hline
                 x & y & direction & angle \\
                 tooLeft & up & left & tooLeft \\
                 tooLeft & up & right & tooRight \\
                 ... & ... & ... & ... \\
                 tooRight & bottom & down & tooRight \\
                 
                \hline
            \end{tabular}
        \end{center}
        
        Note que os exemplos acima são apenas uma pequena amostra das 40 regras
        definidas, que não serão todas apresentadas aqui devido à sua extensão.
        Todas as regras podem ser vistas no arquivo \texttt{remoteDriver.fcl}.

    \subsection*{Método de defuzzificação}
        Defuzzificação transforma os valores de entrada numa variável de saída
        que é interpretada pelo programa para qual a angulação do volante que
        deve estar, partindo do ponto e da direção iniciais. Essa transformação
        é através de alguma transformação numérica, o método aqui utilizado é o
        mais tradicional de defuzzificação: o cálculo do centroide (ou
        baricentro). O método envolve o cálculo do ponto central da função
        \textit{fuzzy} de saída baseando-se nas operações feitas sobre os
        parâmetros de entrada, tal ponto indica a angulação do volante que o
        caminhão irá fazer.
        
    \subsection*{Problemas encontrados}
        Dependendo dos parâmetros $x$, $y$ e do ângulo com o qual o programa é
        iniciado, é possível encontrar pontos em que o programa simplesmente
        não consegue responder com as direções para que o caminhão estacione
        corretamente. Em alguns pontos, o caminhão consegue estacionar, outras
        vezes, ele acaba indo mais para a esquerda ou mais para a direita,
        errando o alvo, mas ainda assim chegando bem próximo ao esperado. 
        
        Como são poucos casos em que isso ocorre, acreditamos que se trata
        apenas de uma questão simples de probabilidade proporcionado pela
        lógica \textit{fuzzy} e, portanto, acreditamos que é um problema
        aceitável dado a natureza probabilística deste trabalho. As condições
        em que o caminhão estaciona erroneamente acontece com mais frequência
        nas regiões de baixo, visto que o caminhão não consegue espaço o
        suficiente para conseguir virar em direção ao estacionamento.
        
\end{document}
